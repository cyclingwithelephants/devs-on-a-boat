\section{Internet connectivity}

Marine internet connectivity is a solved problem and implementing a solution means following established best practices. This is getting better and better.

My strategy in getting good internet is going to involve determining the performance floor such that my internet connection will always be better than calculated. It turns out that the performance floor can be extremely high, with a higher bandwidth and identical latency to my home broadband connection. That's right, I intend to \textit{improve} my internet connection by moving onto a sailboat.

To provide context for this discussion, I've been working from home with my housemate as many have during the pandemic. We've been doing so using an internet connection with 70Mb/s down, 18Mb/s up and 12ms latency. My housemate in particular is a consultant and spends approximately 6 hours a day on zoom - he's responsible for about 70\% of our bandwidth usage (as determined by my routers QoS measurements). As such, meeting these stats will far exceed my need and in theory a download speed above 21Mb/s would be sufficient to work from, I am however going to engineer a solution capable of exceeding my current speeds in the interests of quality of life.

Additionally, I spent a month during the summer of 2020 experimenting with working portably. During this time I was using a £20 4GEE WiFi Mini dongle to great success. I did not measure my connection at this point but research suggests I was achieving around 25Mb/s down and with approximately 50ms latency in rural York \cite{york-wifi}, Even at a location with notably poor 4G performance \cite{poor-york-4g}  this proved to be sufficient for video calls.

As a bar, I am assuming that I would need to be able to have a video call at any point in time, this can be achieved multiple times over with speeds of up to 100Mb/s at distances of up to 22 miles from shore using a single 4G connection \cite{coastal-marine-wifi}. There is no reason for me to travel even 3 miles off-shore except when travelling  between marinas or when doing ocean voyages - I would only be doing these activities during my own time, either weekends or agreed holiday.

\subsection{redundancy and availability}

The solution is built to be redundant at multiple levels, achieving a highly available internet connection which I describe below.

I discuss power availability in its own section.

\subsubsection{Connection types}
Whenever possible I would take advantage of long range wifi, this extends up to 7 miles off-shore in perfect conditions but can more reasonably be expected to be around half this distance under poor conditions \cite{long-range-wifi}. This will of course provide sufficient internet access, it's an otherwise standard WiFi connection. When this is not available I would fallback automatically to a load-balanced 4G connection (5G if available).

\subsubsection{4G}

I anticipate that whenever conventional Wifi is not available my primary internet connection will be a load-balanced 4G connection, meaning I would have up to double the bandwidth of a single 4G connection but with identical latency.

This also provides provider-level redundancy as I would be able to take advantage of competing, distinct networks. This has the added benefit of my 4G coverage being better than that of any single 4G provider.

These considerations (along with hardware considerations) ensure that the 4G connection itself is highly available, which would be imperative in situations where WiFi was not available. 

\subsubsection{Hardware}
By virtue of using multiple, independent pieces of hardware to handle each connection type I am insulated from any single piece of hardware failing from causing an outage.

Additionally, 4G load-balancing will be dealt with by two independent pieces of hardware rather than a single unit with multiple sim slots. This is necessary for hardware level redundancy, ensuring high availability of the 4G connection.

\subsection{5G}

\subsection{Satellite internet}
Starlink is a game changing proposition, with \textit{current} beta performance advertised as "better than nothing" in North America providing a comparable connection to what I get at home. By the time I expect to be in the water, this level of connection will be improved to cover the vast majority of the inhabited world and improve connections such that I would have identical latency and download/upload speeds multiple times that of my current connection \cite{starlink-FAQ}.

Additionally, Coastal areas are expected to have particularly good performance due to anticipated server load being significantly lower at sea than on land. Solving the problem of home-schooling 5 children \textit{whilst on an ocean voyage} has been described by Elon Musk as "relatively easy", with spaceX having filed to use this technology themselves for their autonomous spaceport droneships \cite{starlink-ocean-feasibility}.

To be clear, this would likely not be a possible solution immediately upon moving aboard; Instead I hope to highlight that we can already exceed "good enough" with the trivialisation of broadband-level internet access at a worldwide scale being imminent. If I were writing about internet access 1 year from now it would be \textit{significantly} shorter.